 
\documentclass[12pt,titlepage]{article}

\usepackage[encapsulated]{CJK}
\usepackage{ucs}
\usepackage[utf8x]{inputenc}
\usepackage{multicol}
\usepackage{lipsum}
\usepackage{parallel}
\usepackage[a4paper,margin=1.5cm,footskip=.5cm]{geometry}

\title{\textbf{\begin{CJK*}{UTF8}{gbsn}离骚\end{CJK*} LÍ SĀO INCONTRO AL DOLORE, POEMA DI \begin{CJK*}{UTF8}{gbsn}屈原\end{CJK*} QŪ YUÁN} }
\author{Francesca Godi}
\date{}

\begin{document}
	\maketitle
\section{Autore}
\begin{CJK*}{UTF8}{gbsn}
屈原
\end{CJK*}
\underline{Qū Yuán}(ca. 339-ca. 278 a.C.\footnote{Wilt Idema e Lloyd Haft, Letteratura cinese (Cafoscarina, 2000), 46.}) nacque nel periodo dei principati combattenti (secoli V-III a.C.). Egli apparteneva a una famiglia principesca e servì il sovrano di Chu, ovvero il 
\begin{CJK*}{UTF8}{gbsn}楚怀王\end{CJK*} 
Chǔ Huáiwáng, come 
\begin{CJK*}{UTF8}{gbsn}左徒 \end{CJK*}
zuǒtú, ministro di sinistra.\footnote{Lionello Lanciotti, Letteratura cinese (ISIAO, 2007), 60.} Allora il regno di 
\begin{CJK*}{UTF8}{gbsn}楚 \end{CJK*}
Chǔ era impegnato a contenere le mire espansionistiche del regno di 
\begin{CJK*}{UTF8}{gbsn}秦\end{CJK*} 
Qín, e 
\begin{CJK*}{UTF8}{gbsn}屈原\end{CJK*}
 Qū Yuán consigliò al re 
\begin{CJK*}{UTF8}{gbsn}怀\end{CJK*} Huái di non accettare le offerte di pace del nemico; il partito moderato, per convincere il sovrano diversamente, screditò \begin{CJK*}{UTF8}{gbsn}屈原 \end{CJK*} Qū Yuán in ogni modo, facendolo infine allontanare da palazzo. Dopo il saccheggio della capitale di \begin{CJK*}{UTF8}{gbsn}楚 \end{CJK*} Chǔ fu richiamato a palazzo, ma non riuscì ad avere la fiducia del nuovo sovrano, cadde dunque nuovamente in disgrazia e venne esiliato definitivamente.\footnote{Yuan Chu,Li Sao: incontro al dolore (Lubrina, 1989),11–12.} Quindi si dedicò alla poesia, in cui espresse tutto il suo dolore per il fallimento come ministro, finché il quinto giorno della quinta luna del calendario lunare del 278 a.C. si suicidò nel fiume \begin{CJK*}{UTF8}{gbsn}汨罗 \end{CJK*} Mìluó, nello \begin{CJK*}{UTF8}{gbsn}湖南 \end{CJK*} Húnán.\footnote{Lanciotti, Letteratura cinese, 60.} La storicità di questa figura è stata messa in dubbio alcune volte, ma è oggi generalmente accreditata dagli studiosi sia cinesi che occidentali.\footnote{Masi, \emph{Cento trame di capolavori della letteratura cinese} (Rizzoli,1991), 76.}Egli è stato assunto fin dall'antichità come simbolo dell'uomo di Stato al servizio del popolo, martire per la sua onestà e la sua generosità.\footnote{Ibid., 78.}
 Anche in suo onore è la festa tradizionale sui fiumi, detta festa della barca-dragone, che viene celebrata il giorno della sua morte.\footnote{Chu, \emph{Li Sao: incontro al dolore}, 12.}
\section{Opera}
Il 
\begin{CJK*}{UTF8}{gbsn}离骚\end{CJK*}
 Lí Sāo è uno dei più antichi poemi cinesi, ed è incluso nei 
 \begin{CJK*}{UTF8}{gbsn}楚辞 \end{CJK*}
 Chǔcí (Canti di Chu). Si tratta del lamento di un amministratore che viene licenziato e ingiustamente bandito da corte a causa dei calunniatori e in questo modo sottratto alla beatitudine, poiché la vera vocazione di un gentiluomo era la carriera burocratica.\footnote{Idema e Haft, \emph{Letteratura cinese}, 46–47.} E' la prima opera in versi attribuita ad un singolo autore.\footnote{Masi, \emph{Cento trame di capolavori della letteratura cinese}, 76.}
I 
\begin{CJK*}{UTF8}{gbsn}楚辞\end{CJK*}
 Chǔcí, di cui fa parte il 
 \begin{CJK*}{UTF8}{gbsn}离骚\end{CJK*}
  Lí Sāo, sono stati compilati da 
 \begin{CJK*}{UTF8}{gbsn} 刘向\end{CJK*}
   Liú Xiàng (75-5 a.C.) e arricchiti con il commento e l'opera di 
  \begin{CJK*}{UTF8}{gbsn} 王逸 \end{CJK*}
   Wáng Yì (m. 158 d.C.), che l'ha pubblicata nella sua composizione attuale.\footnote{Lanciotti, \emph{Letteratura cinese}, 59.} Le parti più antiche dei 
  \begin{CJK*}{UTF8}{gbsn} 楚辞\end{CJK*}
    Chǔcí sono testi religiosi, sciamanici, con una forte componente erotica, risalenti al IV s. a.C.\footnote{Idema e Haft, \emph{Letteratura cinese}, 111.} Probabilmente sono stati composti quando ormai le pratiche sciamaniche non erano più in uso, ma conservano le tematiche e la forza simbolica ed emotiva degli originali. In particolare i canti sciamanici parlavano di evocazioni di anime di defunti o di spiriti e di divinità, o di viaggi immaginari verso luoghi ai confini dell'ultraterreno.\footnote{Masi, \emph{Cento trame di capolavori della letteratura cinese}, 77.}
Il 
\begin{CJK*}{UTF8}{gbsn}离骚\end{CJK*}
 Lí Sāo conta 374 versi e il poeta parla in prima persona, poiché l'opera ha un carattere autobiografico.\footnote{Lanciotti, \emph{Letteratura cinese}, 60.}
\begin{CJK*}{UTF8}{gbsn}屈原 \end{CJK*}
Qū Yuán inizia decantando le sue virtù e le sue qualità. Passa poi a raccontare di intrighi e calunnie che gli fecero perdere la fiducia del suo sovrano. Successivamente vi è il lamento per l'ingiustizia subita, con vari riferimenti storici e mitologici. Nella seconda parte,
 \begin{CJK*}{UTF8}{gbsn}屈原\end{CJK*}
  Qū Yuán narra del suo viaggio ( immaginario) verso il cielo, portato da un dragone bianco e da una fenice. Egli cerca un dio, delle dee, ma senza successo; consulta poi un'indovina e un indovino e riprende il viaggio verso l'estremo occidente. Proprio quando sta per salire al cielo più alto, vede il suo paese e gli manca la forza per proseguire:
  \newpage
\begin{multicols}{3}
 \begin{CJK*}{UTF8}{gbsn}
 \hspace{-1.5em}僕夫悲馬懷兮\\
 蜷扃顧而不行\\
 亂日\\
 已矣哉無人-\\
 莫我知兮\\
 又何懷乎故都。\\
 既莫足與為美政兮 \\
吾將從彭咸之所居\footnotemark

\end{CJK*}
\columnbreak

 \begin{CJK*}{UTF8}{gbsn}
\hspace{-1.5em}仆夫悲马怀兮\\
 蜷扃顾而不行\\
 乱日\\
已矣哉无人\\
莫我知人\\
 又何怀乎故都\\
既莫足与为美政兮 \\
吾将从彭咸之所居\\
\end{CJK*}
\columnbreak

\hspace{-1.5em}pú fū bēi mǎ huái \\
quán jiōng gù ér bùxíng\\
luàn rì	\\
yǐ yǐ zāi wúrén	\\
mò wǒ zhī rén	\\
yòu hé huái hū gùdū\\
jì mò zú yǔ wèi měi zhèng xī\\
wú jiāng cóngpéng xián zhī suǒjū\\
\columnbreak
\end{multicols}
\footnotetext{Chu, \emph{Li Sao: incontro al dolore}, 92; 94.}
\vspace{-1.5em}
\begin{center}
Manca il cuore al cocchiere, i cavalli esitano,\\
Volgono la testa e rifiutano di procedere.\\
Congedo\\
Basta! Paese senza uomini, chi potrebbe capirmi?\\
Perché tendere ancora alla città natale?\\
Se non posso operare per il buon governo,\\
raggiungerò Péng Xián là dove egli sta.\footnote{Masi, \emph{Cento trame di capolavori della letteratura cinese}, 77–78.}\\
\end{center}
Il linguaggio poetico nel 
 \begin{CJK*}{UTF8}{gbsn}
 离骚 
 \end{CJK*} 
 Lí Sāo subisce una chiara influenza dei canti sciamanici. La forma è spesso oscura poiché ne riprende allegorie complesse, che impiegano nomi di fiori e piante di cui ora ignoriamo la simbologia. Secondo l'interpretazione tradizionale si tratta della ricerca frustrata di un sovrano giusto dopo le delusioni subite, finché il poeta disperato dall'infruttuosità della sua ricerca decide di annegarsi nel fiume, seguendo l'antico esempio di un leggendario 
 \begin{CJK*}{UTF8}{gbsn}彭咸 \end{CJK*} Péng Xián. Molti passi del \begin{CJK*}{UTF8}{gbsn}离骚 \end{CJK*} Lí Sāo sono quasi incomprensibili, ma si è comunque trascinati dalla potenza delle immagini e dalla libertà della fantasia dell'autore.\footnote{Ibid., 78.} Il sovrano viene visto come un oggetto d'amore, irraggiungibile ma ardentemente desiderato, con una forte connotazione erotica ripresa sempre dai canti sciamanici. Il ministro, ovvero il poeta, assume invece il ruolo del corteggiatore impaziente.\footnote{Idema e Haft, \emph{Letteratura cinese}, 46.}
Dal punto di vista metrico manca la divisione in stanze. I versi sono di sette sillabe (anche se vi sono delle irregolarità), quelli pari sono in rima, mentre quelli dispari sono seguiti dalla sillaba priva di significato 
\begin{CJK*}{UTF8}{gbsn}兮 \end{CJK*}
xī, che ha valore di cesura. Essa era anticamente pronunciata *ghiei e dava un ritmo alla lettura, molto importante quando la poesia era cantata e accompagnata dalla musica (probabilmente il flauto). Si fa inoltre uso di un verso di sei sillabe la cui terzultima sillaba è atona, spesso una particella grammaticale. Questo metro permette di cimentarsi in narrazioni e descrizioni articolate e discorsive, di tono epico-lirico.\footnote{Masi, \emph{Cento trame di capolavori della letteratura cinese}, 77.}$^{,}$\footnote{Idema e Haft, \emph{Letteratura cinese}, 112–113.}
Il termine 
\begin{CJK*}{UTF8}{gbsn}骚 \end{CJK*} sāo significa “pena, “dolore”, ma con il  
\begin{CJK*}{UTF8}{gbsn}离骚 \end{CJK*} Lí Sāo ha finito per indicare una forma poetica il cui nome viene tradotto, benché vi siano solo alcune somiglianze, con “elegia”. I poemetti 
\begin{CJK*}{UTF8}{gbsn}骚 \end{CJK*} sāo sono fatti di versi accoppiati di sei sillabe, con l'aggiunta di 
\begin{CJK*}{UTF8}{gbsn}兮 \end{CJK*} xī al primo dei due.\footnote{Masi, \emph{Cento trame di capolavori della letteratura cinese}, 77.} Dallo stile 
\begin{CJK*}{UTF8}{gbsn}骚 \end{CJK*} sāo ha poi avuto origine il 
\begin{CJK*}{UTF8}{gbsn}赋 \end{CJK*} fù, un componimento in prosa ritmata caratterizzato dall'opposizione semantica dei due segmenti che formano il periodo.\footnote{Chu, \emph{Li Sao:incontro al dolore},17.}
\section{Contesto storico-culturale}
\begin{CJK*}{UTF8}{gbsn}屈原 \end{CJK*}
Qū Yuán visse tra il IV e il III secolo a.C., nell'epoca degli stati combattenti, in cui si fecero più aspre le lotte fra i principali sette stati della Cina feudale per ottenere la supremazia del paese.\footnote{Lanciotti, \emph{Letteratura cinese}, 60.}Questi regni sono 
\begin{CJK*}{UTF8}{gbsn}燕 \end{CJK*} Yān, 
\begin{CJK*}{UTF8}{gbsn}韩 \end{CJK*} Hán, 
\begin{CJK*}{UTF8}{gbsn}齐 \end{CJK*} Qí, 
\begin{CJK*}{UTF8}{gbsn}秦 \end{CJK*} Qín,
\begin{CJK*}{UTF8}{gbsn} 魏 \end{CJK*} Wèi, 
\begin{CJK*}{UTF8}{gbsn}赵 \end{CJK*} Zhào e 
\begin{CJK*}{UTF8}{gbsn}楚\end{CJK*} Chǔ.\footnote{Chu, \emph{The Li Sao: an elegy on encountering sorrows},13.}
Questo periodo di conflitti ebbe origine con la crisi della società nobiliare rivelatasi tra il VI e il V secolo a.C. con le lotte trale  famiglie dell'alta nobiltà e le varie disposizioni che concentravano il potere nelle mani dei capi di regni o principati. A questo processo si coniuga un certo espansionismo militare che porterà alla creazione di uno stato centralizzato sotto il potere di un imperatore 
\begin{CJK*}{UTF8}{gbsn}
秦\end{CJK*} Qín nel 221 a.C. Il rafforzamento del potere centrale attirava verso le corti principesche e agli ambienti dei ministri gruppi di clienti, piccoli gentiluomini detti 
\begin{CJK*}{UTF8}{gbsn}宾客 \end{CJK*} bīnkè (ospiti) o
 \begin{CJK*}{UTF8}{gbsn}舍人\end{CJK*} shèrén (gente di casa). Un celebre consigliere, conosciuto per i clienti che aveva saputo attirare era proprio 
\begin{CJK*}{UTF8}{gbsn}屈原 \end{CJK*} Qū Yuán del regno di 
\begin{CJK*}{UTF8}{gbsn}楚 \end{CJK*} Chǔ.\footnote{Gernet, \emph{Il mondo cinese: dalla prime civilta alla Repubblica popolare}, 56–57; 67.}
Quel periodo caratterizzato da inquietudine, disordine e sconvolgimenti sociali fu terreno fertile per lo sviluppo di scuole di pensiero e di correnti religiose come quelle dei 
\begin{CJK*}{UTF8}{gbsn} 法家\end{CJK*} fǎjiā, o legalisti, di 
\begin{CJK*}{UTF8}{gbsn}孟子\end{CJK*} Mèngzǐ, o Mencio, di
\begin{CJK*}{UTF8}{gbsn} 荀子 \end{CJK*} Xúnzǐ e dei\begin{CJK*}{UTF8}{gbsn}道家 \end{CJK*} dàojiā, o taoisti.
I
\begin{CJK*}{UTF8}{gbsn}法家 \end{CJK*} fǎjiā, o legalisti,il cui pensatore di cui abbiamo più informazioni è \begin{CJK*}{UTF8}{gbsn}韩非 \end{CJK*} Hán Fēi, si occuparono principalmente della riflessione sulla direzione e l'organizzazione dello stato, rispondendo alla preoccupazione generale di rafforzare i regni. Compresero che il principio del potere dello stato risiedeva nelle istituzioni politiche e sociali e tentarono di assoggettare lo stato alla legge, obbiettiva, imperativa e generale, con il principe come unica fonte delle pene e degli onori cdeterminanti l'ordine sociale\footnote{Ibid., 80–81.}.
Invece 
\begin{CJK*}{UTF8}{gbsn}孟子 \end{CJK*} Mèngzǐ, Mencio, si richiamava a 
\begin{CJK*}{UTF8}{gbsn}孔丘 \end{CJK*} Kǒng Qiū, Confucio, e sostenne che la virtù fosse una qualità morale accessibile a tutti e che il principe capace di mostrare la virtù degli eroi mitici e dei primi sovrani si sarebbe imposto a tutta la Cina. Egli ritenne che gli uomini fossero la cosa più importante e che la potenza si dovesse basare su generosità e cura del benessere generale.\footnote{Ibid., 85–86.}
Infine \begin{CJK*}{UTF8}{gbsn}荀子 \end{CJK*} Xúnzǐ ha riconosciuto l'origine sociale della morale e ha una visione negativa dell'uomo, violento e egoista (diversamente da pensatori come 
\begin{CJK*}{UTF8}{gbsn}孟子 \end{CJK*} Mèngzǐ). Sono i doveri e le regole di comportamento a insegnare all'uomo ciò che è giusto e ciò che è sbagliato. Il principe attribuendo titoli e gradi esprime l'ordine che assicura il funzionamento della società.\footnote{Ibid., 86–87.}
Dal punto di vista religioso i 
\begin{CJK*}{UTF8}{gbsn}道家\end{CJK*} dàojiā, o taoisti ebbero come scopo la salvezza del singolo attraverso il ritiro e le pratiche di comportamento che permetterebbero di accrescere la propria forza vitale.\footnote{Ibid., 84.}
\section{Parole chiave}
\begin{itemize}
\item \begin{CJK*}{UTF8}{gbsn}屈原\end{CJK*} Qū Yuán: autore del \begin{CJK*}{UTF8}{gbsn}离骚\end{CJK*}  Lí Sāo
\item \begin{CJK*}{UTF8}{gbsn}离骚\end{CJK*}  Lí Sāo: opera di\begin{CJK*}{UTF8}{gbsn} 屈原\end{CJK*} Qū Yuán
\item \begin{CJK*}{UTF8}{gbsn}楚怀王\end{CJK*} Chǔ Huáiwáng: re di \begin{CJK*}{UTF8}{gbsn}楚\end{CJK*}  Chǔ
\item \begin{CJK*}{UTF8}{gbsn}楚\end{CJK*}  Chǔ: regno di \begin{CJK*}{UTF8}{gbsn}楚怀王\end{CJK*} Chǔ Huáiwáng
\item \begin{CJK*}{UTF8}{gbsn}左徒 \end{CJK*} zuǒtú:  consigliere di sinistra, ruolo di \begin{CJK*}{UTF8}{gbsn}屈原\end{CJK*} Qū Yuán
\item \begin{CJK*}{UTF8}{gbsn}楚辞 \end{CJK*} Chǔcí: raccolta in cui è contenuto il \begin{CJK*}{UTF8}{gbsn}离骚\end{CJK*} Lí Sāo
\item\begin{CJK*}{UTF8}{gbsn} 兮\end{CJK*}  xī: sillaba priva di significato con valore di cesura
\item\begin{CJK*}{UTF8}{gbsn} 骚 \end{CJK*} sāo: forma poetica che ha avuto origine dal\begin{CJK*}{UTF8}{gbsn} 离骚\end{CJK*}  Lí Sāo
\item \begin{CJK*}{UTF8}{gbsn}秦 \end{CJK*} Qín: regno rivale di\begin{CJK*}{UTF8}{gbsn} 楚 \end{CJK*} Chǔ
\item \begin{CJK*}{UTF8}{gbsn}宾客 \end{CJK*} bīnkè: ospiti, ovvero clienti per cui \begin{CJK*}{UTF8}{gbsn}屈原\end{CJK*} Qū Yuán era rinomato
\end{itemize}

\renewcommand\refname{Bibliografia}
\begin{thebibliography}{9}
\bibitem{1}
 Chu, Yuan. \emph{Li Sao: incontro al dolore}. Lubrina, 1989.
\bibitem{2}
—————. \emph{The Li Sao: an elegy on encountering sorrows}. Cheng Wen publ. comp., 1974.
\bibitem{3}
Gernet, Jacques. \emph{Il mondo cinese: dalla prime civilta alla Repubblica popolare}. G. Einaudi, 1978.
\bibitem{4}
Idema, Wilt, e Lloyd Haft. \emph{Letteratura cinese}. Cafoscarina, 2000.
\bibitem{5}
Lanciotti, Lionello. \emph{Letteratura cinese}. ISIAO, 2007.
\bibitem{6}
Masi, Edoarda. \emph{Cento trame di capolavori della letteratura cinese}. Rizzoli, 1991.
\end{thebibliography}

\end{document}